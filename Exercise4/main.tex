\documentclass[11pt, english]{../Template/NTNUoving}
\usepackage[utf8]{inputenc}
\usepackage[T1]{fontenc}
\usepackage{float}
\usepackage{enumitem}
\usepackage{csquotes}
\usepackage{algorithm}
\usepackage{algorithmic}
\usepackage{listings}
\usepackage{listings}
\usepackage{color}
\usepackage{biblatex}
\usepackage{hyperref}
\usepackage{pdfpages}
\ovingnr{1}    % Nummer på innlevering
\semester{Spring 2021}
\fag{Methods in Artificial Intelligence \\ TDT4171}
\institutt{Department for Computer Science}

\begin{document}
\begin{oppgave}
    \begin{punkt}
        The continuous variables are $Age$, $Name$, $Ticket$, $Cabin$, $Fare$ as these are pretty much
        unique for almost all (except for $Cabin$ where more people can live in same cabin, but this does not warrant enough discrete cases to make it discrete). Both $SibSp$ and $Parch$ can be treated as
        both as for most persons this number should be somewhere between 0-10. However, there are edge cases and this is in reality a continuous variable. This means that when we train it
        with these as discrete attributes there is a chance that the DT meets a unhandled case (for example $SibSp = 11$) when used on testing data.
        In my DT model I have handled this by making an educating guess if $Parch$ or $SibSp$ are chosen as discrete attributes and there is no known case of this from the
        training data. This proved to be smart as the highest accuracy was gotten when using $SibSp$ as a discrete attribute (see results below).

        When trained on all discrete attributes these were the results:
        \clearpage
        %test2

        This model is extremely complicated and got an accuracy of 86.8\%.

        The attribute $Embarked$ shouldn't really affect the survival of the passenger so I decided to drop this. The results:
        \clearpage
        %test3
        The graph got a bit simpler and the accuracy increased to 88.0\%, this makes sense because
        the attribute $Embarked$ was overfitting the problem.

        In fact the best results are gotten by also removing the $Parch$ attribute:
        \clearpage
        %test4
        The graph is now easily readable the accuracy improved to 88.5\%.

    \end{punkt}

    \begin{punkt}
        Support for continuous variables were added by changing the $gain$ function to
        loop over the unique values of the continuous attribute and then find the best split point.
        The results by using the attributes $Age$, $Parch$ and $SibSp$ as continuous attributes:
        \clearpage
        % test5

        The accuracy was 86.6\% which is slighty lower than the results gotten by using discrete values on
        $Parch$ and $SibSp$ and without $Age$.

    \end{punkt}

    \begin{punkt}
        As said the DT from \textbf{a)} performed better than the DT from \textbf{b)}. Personally I believed that
        using the $Age$ attribute would be relevant, especially since they followed the \href{https://en.wikipedia.org/wiki/Women_and_children_first}{Women and children first}
        code of conduct when saving passengers on the Titanic.

        1. Bootstrapping: https://towardsdatascience.com/boosting-the-accuracy-of-your-machine-learning-models-f878d6a2d185
        2. Chi squared pruning (feature selection/feature importance)
    \end{punkt}
\end{oppgave}
\end{document}
