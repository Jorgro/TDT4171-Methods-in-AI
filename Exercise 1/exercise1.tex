\documentclass[11pt, a4paper, english]{../Template/NTNUoving}
\usepackage[utf8]{inputenc}
\usepackage[T1]{fontenc}
\usepackage{float}
\usepackage{enumitem}
\usepackage{csquotes}
\usepackage{algorithm}
\usepackage{algorithmic}
\usepackage{listings}
\usepackage{listings}
\usepackage{color}
\usepackage{biblatex}
\usepackage{hyperref}
\ovingnr{1}    % Nummer på innlevering
\semester{Spring 2021}
\fag{Methods in Artificial Intelligence \\ TDT4171}
\institutt{Department for Computer Science}

\begin{document}

% Problem 1
\begin{oppgave}

    % a
    \begin{punkt}
        \begin{align*}
            P(Siblings \leq 2)
            &= P(Siblings = 0) + P(Siblings = 1) + P(Siblings = 2) \\
            &= 0.15 + 0.49 + 0.27 \\
            &= 0.91
        \end{align*}
    \end{punkt}

    % b
    \begin{punkt}
        \begin{align*}
            P(Siblings > 2 | Siblings \geq 1)
            &= \frac{P(Siblings > 2 \wedge Siblings \geq 1)}{P(Siblings \geq 1)} \\
            &= \frac{P(Siblings > 2)}{P(Siblings \geq 1)} \\
            &= \frac{1-P(Siblings \leq 2)}{1-P(Siblings = 0)} \\
            &= \frac{0.09}{0.85} = \frac{9}{85} \approx 0.1059
        \end{align*}
    \end{punkt}

    % c
    \begin{punkt}

    \end{punkt}

    % d
    \begin{punkt}
    \end{punkt}
\end{oppgave}

% Problem 2
\begin{oppgave}
    \begin{punkt}
    \end{punkt}
\end{oppgave}

% Problem 3
\begin{oppgave}
    \begin{punkt}
    \end{punkt}
\end{oppgave}

% Problem 4
\begin{oppgave}
    \begin{punkt}
    \end{punkt}
\end{oppgave}
\end{document}