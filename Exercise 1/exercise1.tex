\documentclass[10pt, a4paper, english]{../Template/NTNUoving}
\usepackage[utf8]{inputenc}
\usepackage[T1]{fontenc}
\usepackage{float}
\usepackage{enumitem}
\usepackage{csquotes}
\usepackage{algorithm}
\usepackage{algorithmic}
\usepackage{listings}
\usepackage{listings}
\usepackage{color}
\usepackage{biblatex}
\usepackage{hyperref}
\ovingnr{1}    % Nummer på innlevering
\semester{Spring 2021}
\fag{Methods in Artificial Intelligence \\ TDT4171}
\institutt{Department for Computer Science}

\begin{document}

% Problem 1
\begin{oppgave}

    % a
    \begin{punkt}
        \begin{align*}
            P(Siblings \leq 2)
            &= P(Siblings = 0) + P(Siblings = 1) + P(Siblings = 2) \\
            &= 0.15 + 0.49 + 0.27 \\
            &= 0.91
        \end{align*}
    \end{punkt}

    % b
    \begin{punkt}
        \begin{align*}
            P(Siblings > 2 | Siblings \geq 1)
            &= \frac{P(Siblings > 2 \wedge Siblings \geq 1)}{P(Siblings \geq 1)} \\
            &= \frac{P(Siblings > 2)}{P(Siblings \geq 1)} \\
            &= \frac{1-P(Siblings \leq 2)}{1-P(Siblings = 0)} \\
            &= \frac{0.09}{0.85} = \frac{9}{85} \approx 0.1059
        \end{align*}
    \end{punkt}

    % c
    \begin{punkt}
        Let $S_1$, $S_2$ and $S_3$ denote the siblings for each of the friends.

        There are 3 permutations for who can have 3 siblings and the other 0 siblings,
        only 1 permutation where each have 1 sibling and $3\cdot2=6$ permutations for who can have 2 siblings, 1 sibling and 0 sibling
        (we first have 3 choices for the friend with 2 siblings, then 2 choices for the one with 1 sibling).
        This gives us:

        \begin{align*}
            P(S_1 + S_2 + S_3 = 3)
            &= 3P(S_1 = 3)P(S_2 = 0)P(S_3 = 0) \\
            &+ P(S_1 = 1)P(S_2 = 1)P(S_3 = 1) \\
            &+ 6P(S_1 = 2)P(S_2 = 1)P(S_3 = 0) \\
            &= 3\cdot0.06\cdot0.15^2 + 0.49^3 + 6\cdot0.27\cdot0.49\cdot0.15\\
            &= 0.240769
        \end{align*}
    \end{punkt}

    % d
    \begin{punkt}
        Let $S_E$ and $S_J$ denote siblings for Emma and Jacob respectfully. We have:
        \begin{align*}
            P(S_E = 0 | S_E + S_J = 3)
            &= \frac{P(S_E = 0 \wedge S_E + S_J = 3)}{P(S_E + S_J = 3)} \\
            P(S_E = 0 \wedge S_E + S_J = 3) &= P(S_E = 0)P(S_J = 3) \\
            P(S_E + S_J = 3) &= 2 P(S_E = 0)P(S_J = 3) + 2 P(S_E = 1)P(S_J = 2) \\
            \implies P(S_E = 0 | S_E + S_J = 3) &= \frac{P(S_E = 0)P(S_J = 3)}{2 P(S_E = 0)P(S_J = 3) + 2 P(S_E = 1)P(S_J = 2)} \\
            &= \frac{0.15\cdot0.06}{2\cdot0.15\cdot0.06 + 2\cdot0.49\cdot0.27} \\
            &= \frac{5}{157} \approx 0.031847
        \end{align*}
    \end{punkt}
\end{oppgave}

% Problem 2
\begin{oppgave}
    %a
    \begin{punkt}
        Every node can be represented by $2^k$ numbers where $k$ is the number of parent nodes when each variable has a Boolean state.
        This gives us the following:
        \begin{center}
            \begin{tabular}{ |c|c| }
             \hline
             Variable & Numbers needed \\
             \hline
             A  & 1 \\
             \hline
             B &  2 \\
             \hline
             C &  2 \\
             \hline
             D &  2 \\
             \hline
             E &  4 \\
             \hline
             F &  4 \\
             \hline
             G &  2 \\
             \hline
             H &  1 \\
             \hline
             Sum &  18 \\
             \hline
            \end{tabular}
            \end{center}
            As we can see the sum is 18 numbers needed and the statement is thus \textbf{true}.
    \end{punkt}

    % A node X is conditionally independent of its non-descendants given its parents.
    % A node X is conditionally independent of all othr nodes in the network given its Market blanket.
    % Markov blanket: Parents, children and children's parents
    % b
    \begin{punkt}
        \textbf{False}, since both $G$ and $A$ have $E$ as an descendant and we are not given any evidence they are not independent.
    \end{punkt}

    % c
    \begin{punkt}
        \textbf{False}, since $E$ is not given both its parents it can't be conditionally independent from $H$.
    \end{punkt}

    % d
    \begin{punkt}
        \textbf{True}, since $E$ is given both its parents it is conditionally independent from $H$.
    \end{punkt}
\end{oppgave}

% Problem 3
\begin{oppgave}

    % a
    \begin{punkt}

    \end{punkt}

     % b
     \begin{punkt}
     \end{punkt}

      % c
    \begin{punkt}
    \end{punkt}

     % d
     \begin{punkt}
     \end{punkt}
\end{oppgave}

% Problem 4
\begin{oppgave}

    % a
    \begin{punkt}
    \end{punkt}

    % c
    \begin{punkt}
    \end{punkt}

    % c
    \begin{punkt}
    \end{punkt}
\end{oppgave}
\end{document}